\appendix
\appendixpage
\section{Terms That Need More Explanation Then A Footnote}
\subsection{Potential} \label{sec:potential}
Potential is the energy change that occurs when the position of an object changes \cite{potential}. There are many potentials, like electric potential, gravitational potential and elastic 
potential. Let me explain the concept with an example. Say you are walking on a set of stairs in the upwards direction. As your muscles move to bring you one step upwards, energy that is used
by your muscles is converted into gravitational potential. Now imagine you turn around and go downwards instead. Notice how that is easier? That is due to the gravitational potential being 
converted back into energy so your muscles have to deliver less energy to get you down. The potential is usually tied to a force, like the gravitational force.

\subsection{Laplacian Operator} \label{sec:laplace}
The Laplacian operator ($\nabla^2$, sometimes also seen as $\Delta$) has two definitions, one for a vector field and one for a scalar field. The two concepts are not indpendent, a vector field 
is composed of scalar fields \cite{vectorscalarfields}. Let us define a vector field first. A vector field is a function whose domain and range are a subset of the Eucledian $\mathbb{R}^3$ space. 
A scalar field is then a function consisting out of several real variables (meaning that the variables can only take real numbers as valid values). So for instance the circle equation 
$x^2 + y^2 = r^2$ is a scalar field as $x, y$ and $r$ are only allowed to take real numbers as their values. 

With the vector and scalar fields defined, let us take a look at the Laplacian operator. For a scalar field $\phi$ the laplacian operator is defined as the divergence of the gradient of $\phi$
\cite{laplacian}. But what are the divergence and gradient? The gradient is defined in \autoref{eq:gradient} and the divergence is defined in \autoref{eq:divergence}. Here $\phi$ is a vector 
with components $x, y, z$ and $\Phi$ is a vector field with components $x, y, z$. $\Phi_1, \Phi_2$ and $\Phi_3$ refer to the functions that result in the corresponding $x, y$ and $z$ values 
\cite{vectorscalarfields}. Also, $i, j$ and $k$ are the basis vectors of $\mathbb{R^3}$, and the multiplication of each term with their basis vector results in $\Phi_1, \Phi_2$ and $\Phi_3$
respectively. If we then combine the two we get the Laplacian operator, as in \autoref{eq:laplacian scalar}.

\begin{subequations}
    \begin{equation}
        \text{grad } \phi = \nabla \phi = \frac{\delta \phi}{\delta x}i + \frac{\delta \phi}{\delta y}j + \frac{\delta \phi}{\delta z}k
        \label{eq:gradient}
    \end{equation}
    \begin{equation}
        \text{div} \Phi = \nabla \cdot \Phi = \frac{\delta \Phi_1}{\delta x} + \frac{\delta \Phi_2}{\delta y} + \frac{\delta \Phi_3}{\delta z}
        \label{eq:divergence}
    \end{equation}
    \begin{equation}
        \nabla^2 \phi = \nabla \cdot \nabla \phi = \frac{\delta^2 \phi}{\delta x^2} + \frac{\delta^2 \phi}{\delta y^2} + \frac{\delta^2 \phi}{\delta z^2}
        \label{eq:laplacian scalar}
    \end{equation}
\end{subequations}

For a vector field $\Phi$ the Laplacian operator is defined as in \autoref{eq:laplacian vector}. Which essential boils down to taking the Laplacian operator of each function and multiply it by
the basis vector.

\begin{equation}
    \nabla^2 \Phi = (\nabla^2 \Phi_1)i + (\nabla^2 \Phi_2)j + (\nabla^2 \Phi_3)k
    \label{eq:laplacian vector}
\end{equation}

\subsection{Interpolation} \label{sec:interpolation}
Interpolation is a form of estimation, where one has a set of data points and desires to know the values of other data points that are not in the original set of data points\cite{interpolation}. 
Based on the original data points, it is estimated what the values of the new data points will be. There are various forms of interpolation like linear interpolation, polynomial interpolation 
and spline interpolation. The CLAuDE model uses linear interpolation which is specified in \autoref{eq:interpolation}. Here $z$ is the point inbetween the known data points $x$ and $y$. 
$\lambda$ is the factor that tells us how close $z$ is to $y$ in the interval $[0, 1]$. If $z$ is very close to $y$, $\lambda$ will have the value on the larger end of the interval, like 0.9.
Whereas if $z$ is close to $x$ then $\lambda$ will have a value on the lower end of the interval, like 0.1.

\begin{equation}
    z = (1 - \lambda)x + \lambda y
    \label{eq:interpolation}
\end{equation}